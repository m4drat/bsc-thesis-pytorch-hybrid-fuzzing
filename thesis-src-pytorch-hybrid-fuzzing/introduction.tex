\section{Introduction}

Software security is a growing concern in the modern world. Due to the fast-paced advancement of information technologies, software systems have become increasingly numerous and complex. As a result, the occurrence of software vulnerabilities has risen, and there is now a greater demand for secure software development practices.

Memory safety vulnerabilities are a particularly significant concern in software security. They refer to programming errors that can cause a program to access memory in unintended ways, potentially leading to system crashes, data leaks, or even full system compromise. Memory safety vulnerabilities are especially prevalent in large codebases written in memory-unsafe languages such as C and C++.

According to \cite{what-science-can-tell-us-about-c-and-c++-security}, for codebases with more than one million lines of code, at least 65\% of security vulnerabilities are caused by memory safety issues in C and C++. The Chromium project security team also highlights the same point in their report \cite{chromium-project-memory-safety}. This alarming statistic emphasizes the importance of addressing memory safety vulnerabilities in software development. This especially applies to critical software systems, such as operating systems, web browsers, machine learning frameworks, etc.

\subsection{AI and Security}

% What is PyTorch and why do we want to fuzz it?

In recent years, AI (Artificial Intelligence) has emerged as a key technology in many domains, including banking, healthcare, transportation, and more. With the rise of AI-powered applications, there is an increasing need for secure AI models and software systems that can withstand cyber threats, as these systems are often used to make critical decisions that affect human lives.

Of particular interest is the security of frameworks that are used to develop AI systems. Often, these systems are the foundation of AI applications. As such, vulnerabilities in AI frameworks can have a significant impact on the security of applications built on top of them.

One of the most popular AI frameworks is PyTorch \cite{pytorch}. PyTorch is an open-source machine learning framework developed by Meta (formerly Facebook). It is used by many companies and organizations, including Microsoft, Uber, Twitter, and more. Despite its popularity, PyTorch is not immune to security vulnerabilities, especially given that it is written in C++, a memory-unsafe language.

Considering the importance of PyTorch in the AI ecosystem, it is crucial to ensure that PyTorch is secure and robust against cyber threats.

\subsection{Objective}

The objective of this work is to perform a comprehensive security analysis of the PyTorch framework using hybrid fuzzing techniques in order to detect and address any memory safety issues. This work is also aimed at enhancing sydr-fuzz \cite{sydr-cutting-edge-dynamic-symbolic-execution} - a hybrid fuzzing tool developed by ISP RAS, which I used to perform the analysis.
