\documentclass[a4paper,14pt]{extarticle}
\usepackage{geometry}
\usepackage{algorithmicx}
\usepackage[utf8]{inputenc}
\usepackage{csquotes}
\usepackage[T1]{fontenc}
\usepackage[main=english, russian]{babel}   % use russian via \foreignlanguage{russian}{}
\usepackage{amsmath}
\usepackage{amsthm}
\usepackage{amssymb}
\usepackage{fancyhdr}
\usepackage{setspace}
\usepackage{graphicx}
\usepackage{colortbl}
\usepackage{tikz}
\usepackage{pgf}
\usepackage{subcaption}
\usepackage{enumitem}
\usepackage{listings}
\usepackage{indentfirst}
\usepackage[colorlinks,citecolor=blue,linkcolor=blue,bookmarks=false,hypertexnames=true, urlcolor=blue]{hyperref} 
\usepackage{indentfirst}
\usepackage{mathtools}
\usepackage{booktabs}
\usepackage[flushleft]{threeparttable}
\usepackage{tablefootnote}
\usepackage{lstfiracode}
\usepackage{xcolor}
\usepackage[ruled,linesnumbered]{algorithm2e}

\usepackage{minted}
\usemintedstyle{tango}
\renewcommand\theFancyVerbLine{\normalsize\arabic{FancyVerbLine}}
% \usemintedstyle{trac}
% \usemintedstyle{perldoc}
% \usemintedstyle{manni}
% \usemintedstyle{emacs}
% \usemintedstyle{friendly}
% \definecolor{bg}{HTML}{282828}

\usepackage{chngcntr}
\counterwithin{equation}{section}
\counterwithin{table}{section}
\counterwithin{figure}{section}

\lstset{basicstyle=\ttfamily}
\setlist[itemize]{noitemsep, topsep=0pt}
\setlist[enumerate]{noitemsep, topsep=0pt}

% \graphicspath{{graphics/}}

% add bibliography using biblatex
% if if all authors in bibliography are needed => add 'maxbibnames=99' in []:
\usepackage[backend=bibtex, citestyle=numeric]{biblatex}
\addbibresource{bibliography.bib}
\DeclareFieldFormat{labelnumberwidth}{#1\adddot}
\setlength{\biblabelsep}{5pt}

\geometry{left=2.5cm} % left margin
\geometry{right=1.5cm} % right margin
\geometry{top=1.5cm} % top margin
\geometry{bottom=1.5cm} % bottom margin
\renewcommand{\baselinestretch}{1.5} % line spacing

\renewcommand{\theenumi}{\arabic{enumi}} % Make all enumerations to be in form digit.digit
\renewcommand{\labelenumi}{\arabic{enumi}}
\renewcommand{\theenumii}{.\arabic{enumii}}
\renewcommand{\labelenumii}{\arabic{enumi}.\arabic{enumii}.}
\renewcommand{\theenumiii}{.\arabic{enumiii}}
\renewcommand{\labelenumiii}{\arabic{enumi}.\arabic{enumii}.\arabic{enumiii}.}

\begin{document}
\begin{titlepage}
    \newpage

    {\setstretch{1.0}
        \begin{center}
            Federal State Autonomous Educational Institution for Higher Education
            National Research University Higher School of Economics
            \\
            \bigskip
            Information Security \\
        \end{center}
    }

    \vspace{8em}

    \begin{center}
        {\Large BACHELOR'S THESIS}\\
        \textsc{\textbf{
                Research project
                \linebreak
                "Hybrid Fuzzing of the PyTorch Framework"}}
    \end{center}

    \vspace{4em}

    {\setstretch{1.0}
        \hfill\parbox{16cm}{
            \hspace*{5cm}\hspace*{-5cm}Prepared by the student of group 191, 4th year of study,\\
            Larionov-Trichkine Theodor Arsenij\\

            \hspace*{5cm}\hspace*{-5cm}Supervisor:\\
            Petrenko Alexander Konstantinovich
            \\

            \hspace*{5cm}\hspace*{-5cm}Consultant:\\
            Kuts Daniil Olegovich
            \\
        }
    }

    \vspace{\fill}

    \begin{center}
        Moscow 2022
    \end{center}

\end{titlepage}
 % title page
\newpage

{
	\hypersetup{linkcolor=black}
	\tableofcontents
}

\newpage
\section*{Annotation}
\addcontentsline{toc}{section}{Annotation} % add Annotation to Table of Contents

As the number and complexity of software systems continue to increase at a rapid pace, an ever-growing number of these systems are becoming critical to our daily lives.

AI takes this trend to a whole new level by allowing software systems to make decisions that were previously reserved for humans. With these advances in the field of information technologies, it is more important than ever to ensure that critical systems are robust and secure against cyber threats.

In this thesis, we will take a look at the problem of software security and how it can be addressed using automated analysis techniques. We will also improve several aspects of the existing hybrid-fuzzing tools and apply them to the PyTorch framework to detect bugs and vulnerabilities in its code.

\section*{\foreignlanguage{russian}{Аннотация}} % this is how to use russian
\foreignlanguage{russian}{
}

\section*{Keywords}
Hybrid Fuzzing, Program Security, Dynamic Analysis, PyTorch, AI Frameworks Security
\pagebreak

\section{Introduction}

Software security is a growing concern in the modern world. Due to the fast-paced advancement of information technologies, software systems have become increasingly numerous and complex. As a result, the occurrence of software vulnerabilities has risen, and there is now a greater demand for secure software development practices.

Memory safety vulnerabilities are a particularly significant concern in software security. They refer to programming errors that can cause a program to access memory in unintended ways, potentially leading to system crashes, data leaks, or even full system compromise. Memory safety vulnerabilities are especially prevalent in large codebases written in memory-unsafe languages such as C and C++.

According to \cite{what-science-can-tell-us-about-c-and-c++-security}, for codebases with more than one million lines of code, at least 65\% of security vulnerabilities are caused by memory safety issues in C and C++. The Chromium project security team also highlights the same point in their report \cite{chromium-project-memory-safety}. This alarming statistic emphasizes the importance of addressing memory safety vulnerabilities in software development. This especially applies to critical software systems, such as operating systems, web browsers, machine learning frameworks, etc.

\subsection{AI and Security}

% What is PyTorch and why do we want to fuzz it?

In recent years, AI (Artificial Intelligence) has emerged as a key technology in many domains, including banking, healthcare, transportation, and more. With the rise of AI-powered applications, there is an increasing need for secure AI models and software systems that can withstand cyber threats, as these systems are often used to make critical decisions that affect human lives.

Of particular interest is the security of frameworks that are used to develop AI systems. Often, these systems are the foundation of AI applications. As such, vulnerabilities in AI frameworks can have a significant impact on the security of applications built on top of them.

One of the most popular AI frameworks is PyTorch \cite{pytorch}. PyTorch is an open-source machine learning framework developed by Meta (formerly Facebook). It is used by many companies and organizations, including Microsoft, Uber, Twitter, and more. Despite its popularity, PyTorch is not immune to security vulnerabilities, especially given that it is written in C++, a memory-unsafe language.

Considering the importance of PyTorch in the AI ecosystem, it is crucial to ensure that PyTorch is secure and robust against cyber threats.

\subsection{Objective}

The objective of this work is to perform a comprehensive security analysis of the PyTorch framework using hybrid fuzzing techniques in order to detect and address any memory safety issues. This work is also aimed at enhancing sydr-fuzz \cite{sydr-cutting-edge-dynamic-symbolic-execution} - a hybrid fuzzing tool developed by ISP RAS, which I used to perform the analysis.


\section{Software Security Analysis Techniques}

As we have seen in the previous section, software security is a question of paramount importance in the modern world. Due to the increasing complexity of software systems, it is no longer feasible to rely only on manual code reviews and testing to ensure that they are secure. Instead, a variety of automated analysis techniques have been developed to help developers detect and address security vulnerabilities in their software.

The security analysis techniques can be broadly divided into two categories:

\begin{itemize}
	\item Static Analysis
	\item Dynamic Analysis
\end{itemize}

In this section, we will provide an overview of static analysis and then delve into a detailed examination of dynamic analysis techniques.

\subsection{Static Analysis}

A set of techniques known as static analysis involves analyzing the source code of a program without executing it. This approach allows us to detect a wide range of problems in the code, potentially examining all possible execution paths.

Although static analysis tends to be more exhaustive, it suffers a lot from false positives as well as false negatives. Furthermore, static analysis tends to be very slow and resource-intensive, especially for large codebases.

To mitigate these concerns, dynamic analysis is frequently employed in conjunction with static analysis. Although it may not be as comprehensive as static analysis, it allows identifying issues that static analysis may miss.

\subsection{Dynamic Analysis}

Dynamic analysis, also known as fuzzing is one of the most popular techniques for finding bugs and vulnerabilities in software. It involves running a program with various inputs and monitoring its behavior. The goal of fuzzing is to detect error conditions in the program by observing its behavior under different inputs.

Consider example \ref{lst:example1}. This program takes a string as an input and checks if the first four characters are equal to "\textbf{FUZZ}". If they are, the program crashes. Otherwise, it does nothing.

\begin{listing}[htp]
	\centering
	\begin{minipage}{.6\linewidth}
		\begin{minted}[linenos=true, tabsize=4]{c}
void crash(char* buf) {
	if (buf[0] == 'F') {
		if (buf[1] == 'U') {
			if (buf[2] == 'Z') {
				if (buf[3] == 'Z') {
					*(int*)NULL = 0x1337;
				}
			}
		}
	}
}
		\end{minted}
	\end{minipage}
	\caption{Fuzzing example}
	\label{lst:example1}
\end{listing}

The goal of a generic fuzzer would be to automatically find an input that would cause the program to crash.

The simplest way to do so would be to exhaustively test all possible inputs. While this works well in theory and is guaranteed to find the bug, it is not feasible in practice, as the number of possible inputs grows exponentially with the size of the input. For a program that processes a string of 10 characters, where each character can be any of the 127 ASCII characters, the total number of possible inputs is $127^{10} \approx 1.0915 \times 10^{21}$. This number is far too large to be tested in a reasonable amount of time. Instead, a smarter approach is required.

\subsubsection{Fuzzers Overview}

To compensate for the exponential growth of the input space, fuzzers use various techniques to guide the input generation. For example, state-of-the-art, general-purpose fuzzer AFL++ \cite{AFLplusplus-Woot20} uses a technique called \textit{coverage-guided fuzzing} to generate inputs that are more likely to trigger bugs. This technique involves instrumenting the program to collect code coverage information and then using this information to guide the generation of inputs towards unexplored parts of the program.

Another example of input generation techniques used by fuzzers is \textit{grammar-based fuzzing}. This technique involves defining a grammar that describes the structure and syntax of valid inputs for a given program. The fuzzer then generates inputs that conform to this grammar, exploring different paths through the grammar to generate diverse inputs. This technique is used by various fuzzers, including Nautilus \cite{nautilus-grammar-fuzzer}, Superion \cite{superion-grammar-fuzzer}, Gramatron \cite{gramatron-effective-grammar-aware-fuzzing}, and others.

Besides different approaches to input generation, fuzzers are also distinguished by the type of target they are designed to test. For example, Nyx \cite{nyx-hypervisor-fuzzer-usenix21} or kAFL \cite{kafl-usenix17} are fuzzers designed to work on a hypervisor level allowing to fuzz OS kernels, drivers, and other hard-to-test components. On the other hand, AFL++ or LibFuzzer are examples of general-purpose fuzzers.

\subsubsection{Fuzz Testing Algorithm}

While fuzzers might look very different on the surface, they all share the same basic structure and follow a similar algorithm. In the paper \cite{the-art-science-and-engineering-of-fuzzing-a-survey}, the authors present a high-level overview of the fuzzing process.

Omitting some details, the fuzzing process can be summarized as follows:

\begin{enumerate}
	\item Preprocessing - prepare a corpus of inputs, instrument the program to collect coverage information, etc.
	\item Scheduling - select fuzzing strategies, etc.
	\item Input generation - select an input from the corpus, mutate the input, generate new inputs, etc.
	\item Input evaluation - run the program with the input, collect feedback (e.g. coverage information, crashes, etc.)
	\item Continue fuzzing until a stopping condition is met (e.g. a timeout)
\end{enumerate}

To implement the fuzzing process described above, a fuzzing loop can be used as shown in Algorithm \ref{alg:fuzzing-loop}.

\begin{figure}[ht]
	\centering
	\begin{minipage}{.7\linewidth}
		\begin{algorithm}[H]
			\caption{Fuzzing loop}
			\label{alg:fuzzing-loop}
			$queue \gets construct\_queue()$ \\
			\While{should fuzz} {
				$input \gets select\_input(queue)$ \\
				$input \gets mutate(input)$ \\
				$feedback \gets run\_program(input)$ \\
				\If{feedback is crash} {
					$report\_bug(input)$ \\
				}

				\If{feedback is interesting} {
					$queue.push(input)$ \\
				}
			}
		\end{algorithm}
	\end{minipage}
\end{figure}

The algorithm presented in Algorithm \ref{alg:fuzzing-loop} provides a simplified representation of the fuzzing process that allows us to concentrate on specific components of the fuzzer.

The natural modularity of the fuzzing process has proven to be beneficial, as shown by the example of LibAFL \cite{libafl-ccs21}. This fuzzer has taken advantage of this modular design by enabling users to create their custom implementations of individual components, thereby allowing greater flexibility and customization of the fuzzing process to tackle specific challenges or meet particular requirements.

\subsubsection{Individual Fuzzer Components}

To further understand the different techniques used by fuzzers, let's take a look at some papers that focus on individual components of the fuzzing process.

One important component is the mutation engine used to generate new inputs from existing ones. In the paper \cite{redqueen-ndss19}, the authors propose a new mutation strategy called Redqueen, which utilizes feedback from previous executions to build input-to-state correspondence. This allows Redqueen to solve simple comparison-based constraints, such as the one in the Listing \ref{lst:example2}, assuming the input-to-state mapping is one-to-one.

\begin{listing}[htp]
	\centering
	\begin{minipage}{.6\linewidth}
		\begin{minted}[linenos=true, tabsize=4]{c}
if (strcmp(buf, "FuZzing1sC00L") == 0) {
	*(int*)NULL = 0x1337;
}
		\end{minted}
	\end{minipage}
	\caption{Example solvable by Redqueen}
	\label{lst:example2}
\end{listing}

Another important component is the input selection strategy. In the paper \cite{effective-seed-scheduling-for-fuzzing-with-graph-centrality-analysis}, the authors propose a new seed selection strategy called \textit{K-Scheduler}, which uses graph centrality analysis to select seeds that are more likely to increase feature coverage. The authors show that this strategy outperforms other seed selection strategies, such as \textit{Entropic}, or next-best AFL-based seed scheduler \textit{RarePath} by 25.89\% and 4.21\%, respectively.

\subsubsection{Challenges}

In conclusion, fuzzing has become one of the best techniques to find bugs in software. Through extensive research, various techniques have been developed and applied to different components of the fuzzing process, such as mutation engines, input selection strategies, and others. However, there are many challenges that haven't been solved yet. Ranging from the scalability of fuzzing to the quality of the generated inputs, there are many areas that can be improved.

One particularly challenging problem is the generation of inputs that satisfy complex constraints. Even with the most advanced fuzzers, it is still difficult, if not impossible, to generate inputs that satisfy constraints such as the one in Listing \ref{lst:example3}. This happens because the constraints may involve complex arithmetic operations, or other hard-to-resolve dependencies between input values. As a result, traditional fuzzing techniques that rely on random or mutation-based input generation with coverage feedback are not sufficient to solve this problem.

\begin{listing}[htp]
	\centering
	\begin{minipage}{.6\linewidth}
		\begin{minted}[linenos=true, tabsize=4]{c}
void vuln(int key) {
	if (key * 0xa9a57b == 0x1337beef) {
		error();
	}
}
		\end{minted}
	\end{minipage}
	\caption{Example solvable by symbolic execution}
	\label{lst:example3}
\end{listing}

That is where another set of techniques called \textit{Symbolic Interpretation} comes into play.

\subsection{Symbolic Interpretation}

Symbolic interpretation, also known as symbolic execution, aims to solve the problem of generating inputs that satisfy complex constraints, such as the one in Listing \ref{lst:example3}.

Essentially, symbolic execution is a powerful technique that enables us to run a program with symbolic inputs instead of concrete ones. By treating program states as sets of constraints on these inputs, we can systematically explore different paths through the code and generate new test cases that can reveal hidden bugs.

For example, the state of the program in Listing \ref{lst:example3} can be defined by this equation: \mintinline{python}{key * 0xa9a57b = 0x1337beef}. Depending on whether this equation is satisfied or not, we either take the \texttt{true} or the \texttt{false} branch. By solving this equation, we can generate an input that would open up the \texttt{true} branch, and thus trigger the \texttt{error()} function. For this particular example, the input \mintinline{python}{0x1337beef / 0xa9a57b = 0x1d} would satisfy the equation and trigger the error. What is notable, for classical fuzzers, it would require exhaustively testing all possible inputs to find this one, as there is no feedback which would guide the fuzzer towards this input.

Now that we have covered the fundamentals of symbolic execution, let us delve deeper into the various components of the symbolic execution process.

\subsubsection{Symbolic Representation}

Symbolic representation is the initial stage of the symbolic execution process where program variables and inputs are represented as symbolic expressions that can be mathematically evaluated and manipulated.

To effectively build and update a program's symbolic state based on the instruction semantics, it is necessary to symbolically execute machine code instructions while simultaneously updating the symbolic state. A convenient approach is to use a dynamic binary analysis framework, such as Triton \cite{triton-sstic2015}, which provides an API for symbolic execution and allows us to easily build symbolic expressions from machine code instructions.

In the Listing \ref{lst:example4}, we can see an example of how Triton can be used to symbolically execute a program from Listing \ref{lst:example3}, and generate an equation for the conditional jump instruction.

% an input that would trigger the \texttt{error()} function.

\begin{listing}[htp]
	\centering
	\begin{minipage}{.9\linewidth}
		\begin{minted}[linenos=true, tabsize=4, breaklines=true, fontsize=\small]{python}
from triton import *

>>> # Create the Triton context with a defined architecture
>>> ctx = TritonContext(ARCH.X86_64)

>>> # Symbolize data (optional)
>>> ctx.symbolizeRegister(ctx.registers.eax, 'sym_eax')

>>> # Execute instructions
>>> ctx.processing(Instruction(b"\xb9\x7b\xa5\xa9\x00")) # mov ecx, 0xa9a57b
>>> ctx.processing(Instruction(b"\xf7\xe1"))             # mul ecx
>>> ctx.processing(Instruction(b"\x3d\xef\xbe\x37\x13")) # cmp eax, 0x1337beef

>>> # Get the symbolic expression
>>> zf_expr = ctx.getSymbolicRegister(ctx.registers.zf)
>>> print(zf_expr)
(define-fun ref!14 () (_ BitVec 1) (ite (= ref!8 (_ bv0 32)) (_ bv1 1) (_ bv0 1))) ; Zero flag
	\end{minted}
	\end{minipage}
	\caption{Triton API example}
	\label{lst:example4}
\end{listing}

Triton provides a powerful mechanism for interpreting machine code instructions and updating the symbolic state simultaneously. However, it may not be able to handle certain scenarios such as external library calls or complex OS-dependent instructions like \texttt{syscall}. In such cases, it may be necessary to actually run the program and symbolically execute as much as possible, while concretizing the remaining instructions that cannot be symbolically executed.

This approach is commonly used by various symbolic execution engines, such as QSym \cite{qsym-usenix2018} and others. In the case of QSym, the symbolic execution engine simply concretizes the instructions that cannot be symbolically executed, and then continues with the symbolic execution. This approach is called \textit{concolic execution} and is widely used in symbolic execution engines.

% With the power of Triton we have a way to maintain a symbolic state of the program.

\subsubsection{Dynamic Constraints Collection}

A key component of concolic execution is dynamic constraints collection, which is performed using a concrete executor that runs a program with specific inputs and collects constraints on those inputs.

To accomplish this, dynamic binary instrumentation (DBI) frameworks are commonly employed. DBI frameworks allow for program instrumentation and constraint collection on-the-fly, providing a convenient solution to perform dynamic analysis. Popular DBI frameworks, such as Pin \cite{pintool-2005}, DynamoRIO \cite{dynamorio-thesis}, and QEMU \cite{qemu-usenix2005} can be used for this purpose.

Typically, per-instruction callbacks are used in DBI frameworks to collect constraints as the program is executed. When a callback is triggered, the corresponding instruction is examined, and the constraints on the input values are collected. These constraints are then combined to form a path condition that represents all the constraints encountered during execution.

With the constraints collected, we can now solve them and generate new inputs.

\subsubsection{Constraints Solving}

To solve the constraints collected during dynamic analysis, a constraint solver is needed. The solver takes the path condition generated from the collected constraints and generates new input values that satisfy the condition. To solve the constraints, a wide range of SMT solvers can be employed, such as Z3 \cite{z3-smt-solver}, Bitwuzla \cite{bitwuza-smt-comp-2020}, CVC5 \cite{cvc5-smt-solver}, and others.

The efficiency and accuracy of the solver play a crucial role in the performance of concolic execution. In some cases, the solver may not be able to find a solution, or the solution may be too complex and time-consuming to compute. To address these issues, various techniques such as constraint simplification and pruning can be used to simplify the constraints and reduce the solution space.

Once the solver generates new input values, the program can be executed again with the updated inputs, and the process of constraint collection and solving can be repeated. This iterative process continues until all paths have been explored, or until a specific goal, such as reaching a specific code location or triggering a specific vulnerability, is achieved.

To illustrate the process of constraint solving, we can use the example from Listing \ref{lst:example4}. In this example, we symbolically executed the \texttt{mul} instruction and generated a constraint on the \texttt{ZF} flag. We can now solve this constraint using the Triton API, as shown in Listing \ref{lst:example5}. The solution to the constraint is \texttt{sym\_eax = 0x1d}, which is the value that would trigger the \texttt{error()} function.

\begin{listing}[htp]
	\centering
	\begin{minipage}{.7\linewidth}
		\begin{minted}[linenos=true, tabsize=4, breaklines=true, fontsize=\small]{python}
>>> # Solve constraint
>>> ctx.getModel(zf_expr.getAst() == 0x1)
{0: sym_eax:32 = 0x1d}

>>> # 0x1d * 0xa9a57b is indeed equal to 0x1337beef
>>> hex(0x1d * 0xa9a57b)
'0x1337beef'
\end{minted}
	\end{minipage}
	\caption{Triton getModel() API example}
	\label{lst:example5}
\end{listing}

\subsubsection{Benefits}

With the ability to symbolically execute a program and generate new inputs that satisfy a specific condition, a few obvious benefits arise.

First, we can automatically generate inputs that would open up new paths in the program. This allows us to explore different program states and as a result, test different aspects of the program. This is called \textit{path exploration} and is one of the main benefits that symbolic execution provides.

The second benefit is the ability to not only explore different paths, but also to check security invariants. For example, we can check if a specific code location is reachable, or if a specific condition can be satisfied. This is called \textit{security invariant checking} and is another key benefit of symbolic execution. This technique allows us to check if a specific vulnerability can be triggered, for example, if it is possible to generate such an input that would trigger an out-of-bounds access. Consider the example from Listing \ref{lst:example6}.

\begin{listing}[htp]
	\centering
	\begin{minipage}{.4\linewidth}
		\begin{minted}[linenos=true, tabsize=4]{c}
void vuln(uint32_t index) {
	char data[64];
	if (index < 74)
		data[index] = 0x37;
}
		\end{minted}
	\end{minipage}
	\caption{Security invariants checking example}
	\label{lst:example6}
\end{listing}

The bug here is obvious -- the \texttt{data} array is only 64 bytes long, but the condition \texttt{index < 74} checks that the index is less than 74. This means that any index greater than 63 would trigger an out-of-bounds write. With the automatic security invariant checking, we can check if it is possible to generate such an input that would trigger the vulnerability.

One particular example of a tool that implements this technique is Sydr. In the paper \textit{Symbolic Security Predicates: Hunt Program Weaknesses} \cite{sydr-symbolic-security-predicates}, the authors proposed a technique called \textit{symbolic security predicates} that allows for automatic security invariant checking.

\subsubsection{Challenges}

While symbolic execution provides a lot of benefits, it also comes with a handful of challenges.

\newparagraph{Symbolic Memory}

One of the main challenges of symbolic execution is the ability to build a precise symbolic model of the program. While modeling simple program with scalar values is relatively easy, modeling memory with pointer operations introduces a new level of indirection that makes the process much more difficult. Additionally, performance drops significantly due to the substantial increase in formula size and complexity when compared to scalar-only modes.

Another crucial aspect of the challenge is determining approximate or exact boundaries for symbolic memory accesses. In general case, a symbolic memory dereference could access any memory location, which is infeasible to model. Therefore, we must find a way to limit memory access to a specific range. For instance, in the paper \textit{Towards Symbolic Pointers Reasoning in Dynamic Symbolic Execution} \cite{symbolic-pointers-reasoning}, the authors study this problem and explore various techniques to address it. One approach proposed involves the use of heuristics to determine the leftmost boundary by extracting the concrete portion from the abstract syntax tree of the address expression.

\newparagraph{Unsolvable Constraints}

Another challenge arises from the fact that modern SMT solvers are not perfect at solving all SMT problems efficiently. The SMT problem is known to be NP-complete, which means there is no known algorithm that can solve all SMT problems efficiently. As a result, although modern SMT solvers can handle many real-world problems, there are still scenarios where the solver may fail to find a solution or take an excessively long time to do so. This limitation can hamper the performance and effectiveness of the symbolic execution process.

\newparagraph{Incomplete Symbolic Model}

Additional obstacle of symbolic execution is the potential incompleteness of the symbolic model. It may not always be possible or practical to completely model a program symbolically, especially when dealing with interactions with the outside world, such as system calls. This can result in discrepancies between the symbolic program state and the real one, leading to unsolvable constraints that impede the effectiveness of the technique. However, some approaches, such as concolic testing, can help mitigate this challenge by combining symbolic and concrete execution.

\newparagraph{Path Explosion}

Finally, a significant challenge in various types of program analysis, including symbolic execution, fuzzing, and path-sensitive static analysis, is the path explosion problem. This issue arises because the number of control-flow paths in a program increases exponentially with the program's size. As a result, the symbolic execution engine may not be able to explore all the paths within a reasonable timeframe. To overcome this challenge, researchers have proposed various techniques, such as path pruning and constraint prioritization, that aim to reduce the number of explored paths without compromising the completeness of the analysis. However, these techniques have their limitations, and achieving complete path coverage remains a challenging problem in program analysis.

\subsection{Hybrid Fuzzing}

% Symbolic execution is shown to be a powerful technique for program analysis. However, it is not without its limitations. One of the main challenges of symbolic execution is the path explosion problem. As the number of paths in a program increases exponentially with the program's size, the symbolic execution engine may not be able to explore all the paths within a reasonable timeframe. To overcome this challenge, researchers have proposed various techniques, such as path pruning and constraint prioritization, that aim to reduce the number of explored paths without compromising the completeness of the analysis. However, these techniques have their limitations, and achieving complete path coverage remains a challenging problem in program analysis.

One of the first attempts to combine symbolic execution was presented as a \cite{driller-ndss16}

\subsubsection{Examples}


\section{PyTorch Fuzzing}

PyTorch is a popular open-source machine-learning framework that has gained immense popularity in recent years. Developed by Meta (formerly Facebook), PyTorch has emerged as one of the most widely used machine learning frameworks due to its ease of use, flexibility, and dynamic computational graph, making it a popular choice for researchers and developers alike.

PyTorch is a critical component of many applications across various industries such as banking, healthcare, insurance, and many others. It is used for natural language processing, image classification, speech recognition, and other tasks. In banking, PyTorch is used to develop fraud detection systems, while in healthcare, it is used to diagnose diseases and predict patient outcomes. The insurance industry uses PyTorch to analyze risk and predict losses. The flexibility of PyTorch enables it to be used in many other domains as well. PyTorch has been used to build many state-of-the-art machine learning models and is a vital tool in the field of deep learning.

Despite its popularity and usefulness, PyTorch has several challenges that must be addressed to ensure its reliability and security. PyTorch has multiple dependencies, and it includes a considerable amount of C/C++ code which implies that it is susceptible to memory safety vulnerabilities. Moreover, PyTorch is an interesting target for adversaries since it is used in critical applications. Therefore, it is crucial to ensure that PyTorch is secure and free from vulnerabilities. Fuzzing is a valuable technique that can help identify bugs and vulnerabilities in PyTorch, making it more robust and secure. By fuzzing PyTorch, we can ensure that it can withstand attacks and continue to operate correctly in real-world applications.

This chapter delves into the concept of PyTorch fuzzing and its importance in enhancing the reliability and security of PyTorch. The exploration commences with an examination of its attack surface. Subsequently, fuzzing harnesses will be developed for the relevant sections of the codebase. The fuzzing methodology and the outcomes of the work will be outlined as well.

\subsection{Attack Surface Mapping}

To initiate the fuzzing process, it is essential to determine the attack surface of PyTorch. The attack surface refers to all the entry points through which an attacker can potentially interact with the system and launch an attack. In the case of PyTorch, its attack surface includes various modules, libraries, and dependencies that it uses.

To identify interesting parts of the codebase that are relevant to the attack surface, a manual code analysis was conducted. This analysis has highlighted several modules that are particularly interesting to analyze, including the model loading and RPC communications modules.

\subsubsection{Model Loading}

The process of loading pre-trained models is a crucial entry point that attackers can exploit to gain access to the system. This process is typically handled by the model loading module, which can be accessed via the \texttt{torch.load()} function.

During the loading process, the \texttt{torch.load()} function goes through several deserialization steps, also known as unpickling, to recreate the original object from the byte stream. Deserialization is a common source of vulnerabilities in many applications, as it can be difficult to implement correctly. Unfortunately, PyTorch is not immune to this issue. Additionally, since the implementation is written in C++, it is even more susceptible to memory safety vulnerabilities.

The code responsible for model loading and parsing is mostly located in these files:

\begingroup
\begin{itemize}
    \item jit/serialization/import.cpp
    \item jit/ir/irparser.cpp
    \item jit/serialization/unpickler.cpp
    \item jit/runtime/interpreter.cpp
    \item jit/frontend/schema\_type\_parser.cpp
\end{itemize}
\endgroup

\subsubsection{Remote Communications (RPC)}

Besides model loading, PyTorch has another interesting mechanism that attackers can exploit - the RPC communications module.

The RPC module in PyTorch is a complex system that opens up new, remotely accessible attack vectors. PyTorch uses the RPC module to implement distributed training and inference that allows users to train and execute models across multiple machines. This feature is essential for large-scale applications that require high computational power. However, it increases the security risks by creating additional entry points for attackers.

PyTorch uses various types of RPCs such as \texttt{RRef} (Remote Reference),\\ \texttt{ScriptCall}, and others to interact with remote machines. Before sending RPCs, they are serialized into pickled objects using the \texttt{torch::jit::pickle} function. The RPCs are then sent using different backends like TensorPipe, Gloo, and MPI. Once received, the RPCs are deserialized using the \texttt{torch::jit::unpickle} function, and the target \texttt{Message}'s class \texttt{fromMessage()} method is called. This leaves the receiver with a plain message that can be further processed.

Unfortunately, the complexity of this system makes it prone to bugs and vulnerabilities. Multiple serializations and deserializations of messages can introduce bugs, and the fact that the RPC protocol is implemented in C++ makes it an attractive target to look for memory safety vulnerabilities. Moreover, given the memory-unsafe nature of the RPC protocol, a single bug could potentially allow an attacker to execute code remotely on a target machine. As a result, fuzzing the RPC module is highly recommended to identify and address potential vulnerabilities.

Taking that into consideration, the following files have been identified as the most interesting targets for security research.

\begingroup
\begin{itemize}
    \item distributed/rpc/*.cpp
    \item jit/serialization/unpickler.cpp
\end{itemize}
\endgroup

\subsubsection{Finding Fuzz-Targets}

Having identified various sections of the PyTorch codebase that are relevant to the attack surface, we can proceed to the second part of the attack surface mapping - identifying specific functions and methods to fuzz.

To achieve this, two different approaches have been used:

\begin{enumerate}
    \item \textbf{Manual code review} - A manual code review of the PyTorch codebase has been performed to identify relevant functions that are confined to the defined attack surface.
    \item \textbf{CodeQL} - A CodeQL \cite{ql-object-oriented-queries-on-relational-data} has been utilized to broadly search for interesting functions and methods that perform some kind of deserialization or parsing.
\end{enumerate}

The first approach is straightforward and does not require any additional tools. However, it is time-consuming and requires a lot of manual work. Nevertheless, it yields the best results since it allows a researcher to precisely identify the functions that might be interesting to fuzz.

The second approach lacks "precision" but can be automated and scaled to a large codebase. It allows for quick identification of a narrowed-down set of functions that are worth looking into. However, it is not as precise as the first approach since it relies on heuristics and does not "understand" the code. As a result, it can miss some relevant functions. Nevertheless, it is a good starting point for fuzzing since it can help identify interesting functions that can be further analyzed manually.

To begin with, the second approach was employed to pinpoint some specific functions that are worth looking into. A CodeQL query \ref{appendix:codeql-query} was developed to search for functions that have two parameters:

\begin{enumerate}
    \item The first parameter is a pointer to data of "parsable" types. For example - \texttt{char*}, \texttt{byte*}, and others.
    \item The second parameter is an integer that represents the size of the data. For example - \texttt{int}, \texttt{size\_t}, and others.
\end{enumerate}

With that in place, a few more heuristics were added to filter out irrelevant functions. Finally, \textit{Cyclomatic Complexity} \cite{cyclomatic-complexity-density} was used to rank the results and identify the most complex functions. Some results of the query are shown in Table \ref{table:codeql-results}.

\begin{table}[h]
    \centering
    \begin{tabular}{cl}
        \toprule
        \textbf{Complexity} & \textbf{Function}                      \\
        13                  & \texttt{rpc::parseWireSections}        \\
        6                   & \texttt{Unpickler::readSlowWithBuffer} \\
        5                   & \texttt{TokenTrie::insert}             \\
        4                   & \texttt{rpc::wireDeserialize}          \\
        \bottomrule
    \end{tabular}
    \caption{CodeQL query results}
    \label{table:codeql-results}
\end{table}

These findings served as a solid starting point. From here, the functions were manually reviewed, and the codebase was thoroughly studied, employing the first approach.

Finally, a list of the most interesting functions that are worth fuzzing has been compiled. The list is shown in Table \ref{table:fuzz-targets}.

\begin{table}[h]
    \centering
    \begin{tabular}{cl}
        \toprule
        \textbf{Function}                 \\
        \texttt{jit::parseIR}             \\
        \texttt{jit::load}                \\
        \texttt{rpc::deserializeResponse} \\
        \texttt{rpc::deserializeRequest}  \\
        \bottomrule
    \end{tabular}
    \caption{Fuzz targets}
    \label{table:fuzz-targets}
\end{table}

The first two functions are related to the JIT module and are responsible for parsing and loading the pickled data. Some examples of such data are: \texttt{saved models}, \texttt{serialized requests}, and others.

The last two functions are related to the RPC module and are responsible for deserializing RPC requests and responses. These functions are interesting because they are directly processing untrusted data that is received from the network.

Besides that, another interesting function - \\ \texttt{jit::preoptimizeGraph}, has been identified within the JIT module. It has been chosen as a target for differential fuzzing, with the objective of uncovering bugs associated with JIT graph optimizations.

\subsection{Preparing PyTorch for Fuzzing}

Having identified the fuzz targets, the subsequent step involves the development of a fuzzing harness. The goal of the fuzzing harness is to provide a way to feed the fuzz target with data and collect the results of the execution. In alignment with the chosen objective, \textit{LibFuzzer}-compatible \cite{libfuzzer-secdev-2016} targets have been created for each of the functions enumerated in Table \ref{table:fuzz-targets}.

Each libFuzzer-based target shares the same structure. The structure is shown in Listing \ref{listing:fuzz-target-structure}.

\begin{listing}[h]
    \centering
    \begin{minipage}{.9\linewidth}
        \begin{minted}[linenos=true, tabsize=4]{cpp}
int LLVMFuzzerTestOneInput(const uint8_t* data, size_t size) {
    // 1. Prepare the input data
    // 2. Call the fuzz target with the parsed data
    // 3. Return 0
}
    \end{minted}
    \end{minipage}
    \caption{Fuzz target structure}
    \label{listing:fuzz-target-structure}
\end{listing}

The goal of a generic fuzzing harness is to pass the data generated by the fuzzer to the fuzz target and clean up the resources after the execution is finished. The fuzzing harness is also responsible for handling the exceptions that might be thrown by the fuzz target.

The fuzzing harnesses that have been developed for PyTorch are similar to the generic one. However, they also perform some additional steps. For example, they handle some PyTorch-specific exceptions.

The developed fuzzing harnesses are listed below:

\begin{itemize}
    \item \href{https://github.com/ispras/oss-sydr-fuzz/blob/028e36875424aec00ef7375006c413a4be164197/projects/pytorch/irparser_fuzz.cc}{\texttt{irparser\_fuzz.cc}} - fuzzes \texttt{jit::parseIR}
    \item \href{https://github.com/ispras/oss-sydr-fuzz/blob/028e36875424aec00ef7375006c413a4be164197/projects/pytorch/load_fuzz.cc}{\texttt{load\_fuzz.cc}} - fuzzes \texttt{jit::load}
    \item \href{https://github.com/ispras/oss-sydr-fuzz/blob/028e36875424aec00ef7375006c413a4be164197/projects/pytorch/message_deserialize_fuzz.cc}{\texttt{message\_deserialize\_fuzz.cc}} - fuzzes \texttt{rpc::deserializeResponse} and \texttt{rpc::deserializeRequest}
    \item \href{https://github.com/ispras/oss-sydr-fuzz/blob/028e36875424aec00ef7375006c413a4be164197/projects/pytorch/jit_differential_fuzz.cc}{\texttt{jit\_differential\_fuzz.cc}} - fuzzes three related methods - \texttt{jit::parseIR}, \texttt{jit::preoptimizeGraph}, and \texttt{jit::InterpreterState(code).run()} with differential fuzzing
\end{itemize}

However, the development of fuzzing harnesses is only one more step on the road toward hybrid fuzzing. The next step is to compile all the necessary build targets.

\newparagraph{LibFuzzer Target}

The first one is the fuzz target itself. It is the main entry point for the fuzzer. This target is compiled using the \textit{clang} compiler with the \textit{libFuzzer} library linked, as well as the \textit{libasan} library. The latter is known as the \textit{AddressSanitizer} \cite{address-sanitizer-usenix-2012} and is required to maximize the chances of finding memory-related bugs.

To build this target, the following compilation flags were added to the build script: \texttt{-fPIC -g -fsanitize=fuzzer,address,bounds}. To easily distinguish the produced binary from other targets, a \texttt{\_fuzz} suffix was added to its name.

\newparagraph{Sydr Target}

The second type of build target is the plain binary for symbolic execution, which reads the data directly from the file and passes it to the fuzz target, without involving the fuzzer. It allows the symbolic execution engine to execute the fuzz target with concrete data. This target is also compiled with the help of the \textit{clang} compiler, however, the flags are different: \texttt{-fPIC -g}. Moreover, this time sanitizers are not used, as they would only complicate the symbolic execution process. A \texttt{\_sydr} suffix was also added to the name of the binary.

\newparagraph{Coverage Target}

And the last type of build target is the binary that is used to collect the coverage information. It is compiled with the \textit{clang} compiler and the following flags: \texttt{-fPIC -g -fprofile-instr-generate -fcoverage-mapping}. As for the previous targets, the \texttt{\_cov} suffix has been incorporated into the binary's name.

\subsection{PyTorch Hybrid Fuzzing}

With all the artifacts produced, we can now proceed to the next step - hybrid fuzzing. The goal of this step is to prepare the corpus, perform the fuzzing, and analyze the results. To carry out hybrid fuzzing, the \textit{sydr-fuzz} tool, developed by ISP RAS, has been employed. The \textit{sydr-fuzz} framework is a versatile tool for hybrid fuzzing, enabling the seamless integration of symbolic execution engines and state-of-the-art fuzzers like \textit{AFL++} and \textit{libFuzzer} for efficient fuzzing of the target program.

\subsubsection{Preparing the Corpus}

Prior to commencing the fuzzing, it is necessary to prepare the corpus for the fuzzer. The corpus represents a collection of inputs that the fuzzer utilizes to generate new inputs.

As most of the targets were aimed at fuzzing the unpickling functionality, the corpus was gathered by extracting test models from the PyTorch repository. Besides unpickling, a few targets were aimed at fuzzing the IR-parsing functionality. For these targets, tests that use the \texttt{jit::parseIR} method were found and the corresponding intermediate representations (IRs) were extracted. An example of such a test case is shown in Listing \ref{listing:ir-program}.

\begin{listing}[h]
    \centering
    \begin{minipage}{.75\linewidth}
        \begin{minted}[linenos=true, tabsize=4]{cpp}
graph(%a : Tensor):
      %b : Tensor = aten::mul(%a, %a)
      %c : Tensor = aten::mul(%b, %b)
      %d : Tensor = aten::mul(%c, %c)
      %c_size : int[] = aten::size(%c)
      %c_alias : Tensor = aten::view(%c, %c_size)
      %e : Tensor = aten::mul(%b, %d)
      %f : Tensor = aten::mul(%c_alias, %c_alias)
      %output : Tensor = aten::mul(%e, %f)
      return (%output)
    \end{minted}
    \end{minipage}
    \caption{IR program extracted from the PyTorch repository}
    \label{listing:ir-program}
\end{listing}

\subsubsection{Dynamic Analysis Pipeline}

With all the necessary preparations completed, the actual fuzzing process can now be initiated. Fuzzing is carried out through a continuous dynamic analysis pipeline, as depicted in Diagram \ref{fig:dynamic-analysis-pipeline}. This pipeline operates continuously, reflecting the iterative nature of bug-fixing and testing. Whenever a new version of the program is released or a patch is proposed, the pipeline is triggered.

The pipeline consists of the following steps:

\begin{figure}[h]
    \centering
    \includegraphics[width=10cm]{assets/dynamic-analysis-pipeline-tnr.png}
    \caption{Dynamic analysis pipeline}
    \label{fig:dynamic-analysis-pipeline}
\end{figure}

\newparagraph{Hybrid Fuzzing}

The first step of the pipeline is hybrid fuzzing. It is performed with the help of the \textit{sydr-fuzz} tool. The tool takes a toml configuration file as input, which specifies the targets to be fuzzed and other parameters. For example, the configuration file for the \texttt{jit\_differential} target is shown in Listing \ref{listing:jit-differential-config}.

\begin{listing}[h]
    \centering
    \begin{minipage}{.75\linewidth}
        \begin{minted}[linenos=true, tabsize=4, breaklines=true]{toml}
[sydr]
args = "-l debug"
target = "/jit_differential_sydr @@"

[libfuzzer]
path = "/jit_differential_fuzz"
args = "-detect_leaks=0 -rss_limit_mb=30720 -timeout=300 -report_slow_units=350 /ir_corpus"

[cov]
target = "/jit_differential_cov @@"
\end{minted}
    \end{minipage}
    \caption{Configuration file for the \texttt{jit\_differential} target}
    \label{listing:jit-differential-config}
\end{listing}

This configuration file contains three sections: \texttt{sydr}, \texttt{libfuzzer}, and \texttt{cov}. The \texttt{sydr} section contains information about the target program and how to execute it under the symbolic execution engine. The \texttt{libfuzzer} section contains libfuzzer-related information, such as the path to the fuzz target binary and the arguments that should be passed to it. Finally, the \texttt{cov} section contains information about the coverage target.

With the configuration file prepared, the fuzzing campaign can be started. To do so, the \textit{sydr-fuzz} binary needs to be executed as follows: \texttt{sydr-fuzz -c jit\_differential.toml run}. The command will start the fuzzing process and will print the results to the console.

\newparagraph{Corpus Minimization}

After running the fuzzer for some time, the subsequent step in the pipeline involves corpus minimization. The goal of this step is to reduce the size of the corpus by removing the redundant inputs. The corpus minimization is performed using the \textit{sydr-fuzz} tool in the \textit{cmin} mode. Under the hood, \textit{sydr-fuzz} uses either \textit{afl-cmin} or the \textit{libfuzzer -merge=1} utilities.

This step is optional, however, it is highly recommended, because it greatly improves the performance of the next step - security invariants verification. Without corpus minimization, the security invariants verification is less likely to find any bugs, as it will potentially re-execute almost the same inputs multiple times.

\newparagraph{Security Invariants Verification}

The next step of the pipeline is security invariants verification. The idea behind this step is to execute the \textit{sydr} symbolic execution engine in the \textit{security} mode, with inputs from the minimized corpus. The \textit{security} mode is a special mode that is designed to check for violation of various security invariants, such as integer overflows, null pointer dereferences, buffer overflows, and others. If any of the invariants are violated, the symbolic execution engine will report the corresponding bug.

Running the symbolic execution engine in the \textit{security} mode is a very expensive operation. For this reason, efforts have been directed toward optimizing its performance. The results are discussed in chapter \ref{hybrid-fuzzer-improvements:optimizing-security-predicates}.

\newparagraph{Crashes Processing}

Near the end, the crashes processing is performed. During previous steps, \textit{sydr-fuzz} may have found some crashes. At this stage, those crashes need to be deduplicated and preprocessed. To do so, \textit{sydr-fuzz} employs the \textit{casr} tool \cite{casr-cluster-ispras-2021}. This tool performs automatic crashes processing, which includes deduplication, and clusterization.

Executing \textit{sydr-fuzz} in this mode, using the following command: \texttt{sydr-fuzz -c jit\_differential.toml casr}, will produce a set of processed crashes.
\newparagraph{Fuzz-Campaign Assessment}

Finally, after all the steps are completed, the results of the fuzzing campaign are assessed. The assessment is performed manually by the user, and includes the following steps:

\begin{enumerate}
    \item Bug triaging and reporting
    \item Code coverage analysis
\end{enumerate}

During the bug triaging and reporting step, the user needs to analyze the found bugs and decide whether they are real bugs or not. If the bug is proven to be real, the user needs to report it to the developers of the target program.

The code coverage analysis step is performed to assess the effectiveness of the fuzzing campaign. The goal of this step is to help the user to understand whether the fuzzing campaign was effective and whether it is worth continuing it. The code coverage analysis is performed using the \textit{sydr-fuzz} tool in the \textit{cov} mode.

This concludes the description of the dynamic analysis pipeline, which has been used to fuzz the PyTorch framework. In the next section, some of the findings will be discussed.

\subsection{Results Overview}

As a result of this work, multiple bugs were found in different parts of the PyTorch framework. The majority of the bugs were discovered within the module responsible for unpickling. Nevertheless, it is worth noting that at least one remotely-accessible bug was identified in the RPC module.

All discovered bugs were reported to the PyTorch framework developers, who confirmed their validity and subsequently merged the corresponding pull requests into the PyTorch repository. The pull requests related to the bugs are listed below:

\begin{itemize}
    \item \href{https://github.com/pytorch/pytorch/pull/94300}{\#94300: Add size check before calling stack\_.at(dict\_pos) in unpickler.cpp}
    \item \href{https://github.com/pytorch/pytorch/pull/94298}{\#94298: Add stack emptiness checks inside interpreter.cpp}
    \item \href{https://github.com/pytorch/pytorch/pull/94297}{\#94297: Add size check before calling .back() in rpc/script\_call.cpp}
    \item \href{https://github.com/pytorch/pytorch/pull/94295}{\#94295: Add exception handlers for stoll in schema\_type\_parser.cpp}
    \item \href{https://github.com/pytorch/pytorch/pull/91401}{\#91401: Add out-of-bounds checks inside irparser.cpp and unpickler.cpp}
\end{itemize}

In section \ref{results:pytorch-findings}, a closer inspection of the bugs found, and the pull created will be performed.


\section{Hybrid Fuzzer Improvements}

During the fuzzing of the PyTorch, several drawbacks of the hybrid fuzzing solution have been identified. In this section, the objective is to address these concerns with the aim of enhancing the overall performance of the \textit{sydr-fuzz} framework.

\subsection{Scheduling Symbolic Pointers Modeling}

Almost any program written in C/C++ uses pointer operations, which necessitates the modeling of pointers for a precise symbolic representation of the program. However, as was already discussed in section \ref{symbolic_execution:challenges:symbolic_memory}, modeling symbolic pointers is a very computationally expensive process. Nevertheless, it is required to find new paths in the program, that would otherwise be unreachable.

The same applies to PyTorch. It uses pointers extensively, and thus it is required to model them to achieve maximum testing completeness. In fact, it has been observed that symbolic pointer modeling is necessary for almost any project under examination.

\subsubsection{Problem Statement}

With that being said, it is impossible to simply disable symbolic pointer modeling, as it would prevent finding new paths in the program, and thus degrade the overall testing completeness. On the other hand, it is not feasible to enable it all the time, as it would significantly slow down the fuzzing process.

So, the objective is to devise a scheduling strategy, that allows to achieve the maximum possible code coverage, without sacrificing performance.

In this context, a scheduling strategy refers to a mechanism that enables running Sydr with memory modeling enabled only when it provides the most significant advantages.

\subsubsection{Proposed Solution}

To accomplish this, a straightforward yet efficient scheduling strategy has been implemented, whereby Sydr with memory modeling enabled is executed once in every N runs.

This strategy "exploits" how caching mechanism works for already traversed branches. By allowing Sydr to execute multiple times before running it with memory modeling enabled, the cache saturates with the branches that are not dependent on the symbolic pointers. This way, when Sydr is eventually executed with memory modeling enabled, it will be able to reuse most of the results from previous runs, having to perform the expensive symbolic pointers modeling only for a small portion of new branches. This significantly improves the performance of the symbolic pointer modeling, enabling the discovery of new paths in the program, while still maintaining a reasonable execution time.

\subsubsection{Experimental Evaluation}

After implementing the scheduling strategy, the optimal value of N had to be determined. The value of N should be large enough to not significantly degrade the overall performance of the hybrid fuzzer, and at the same time allow the cache to saturate. However, it should not be excessively large, as this would hinder the discovery of new paths in the program.

To determine the optimal value of N, multiple experiments were conducted using the \textit{FuzzBench} benchmarking platform \cite{fuzzbench-2021}. The results of these experiments are discussed in section \ref{results:symbolic-pointers-modeling-scheduling}.

Based on the outcomes of these experiments, the decision was made to set N=25 as the default value for the scheduling strategy implemented in \textit{sydr-fuzz}.

A new configuration parameter called \texttt{symaddr} has been introduced, allowing users to customize the number of runs performed between symbolic pointer modeling runs. This parameter can be used to fine-tune the performance of the \textit{sydr-fuzz} for a specific target.

\subsection{Enhancing Security Invariants Checking Mechanism} \label{hybrid-fuzzer-improvements:optimizing-security-predicates}

The next issue that was identified during the fuzzing of the PyTorch is related to the security predicates mechanism. As was already mentioned in section \ref{hybrid-fuzzing:sydr}, \textit{Sydr} has a mechanism that allows it to check for various security invariants violations, such as buffer overflows, null pointer dereferences, integer overflows, and others. Unfortunately, security predicates often tend to produce false positives. That is why \textit{sydr-fuzz} implements automatic verification of security predicates results.

\subsubsection{Security Predicates - Violations Verification}

To verify that the bug is real, \textit{sydr-fuzz} executes the target program built with sanitizers, on the input generated by \textit{security predicates} mechanism. When a sanitizer reports an error in the same location that was identified as the error source by security predicates, the bug is considered to be verified. In other words, if the error is detected in the expected location, it confirms that the security predicates accurately captured the potential vulnerability and that the input seed triggered the vulnerability.

To perform the validation, \textit{sydr-fuzz} needs to symbolize \textit{Sydr}'s log file, which contains the information about the location of the error. Originally, \textit{sydr-fuzz} used a Python script that relied on \textit{addr2line} and \textit{objdump} tools to perform this operation. However, it was discovered that this step is highly time-consuming and negatively impacts the overall performance of the hybrid fuzzer. That is why the decision was made to optimize the performance of this step by implementing it in Rust. This optimization resulted in a significant improvement in the performance of the verification process.

\subsubsection{Utilizing Debug Information to Improve Annotation Speed}

Before the development of the module could start, it was necessary to gain a comprehensive understanding of how the debug information is stored and how it could be interacted with from Rust.

\newparagraph{DWARF Debug Information Format}

The DWARF debugging information format is a standard way to store debugging information in object files. It is used by many compilers and debuggers, including \textit{GCC}, \textit{Clang}, and others. The DWARF format is designed to be extensible, efficient, and language-independent. It uses a series of data structures to describe the program in a way that is both human-readable and machine-readable. Debug information can be used to partially reconstruct the source code, even when only the binary file is available. By leveraging this information, it is possible to recover variable names, file names, function names, and other important entities that were present in the original code.

That is precisely the information that is required for the annotation process. With its help, it is possible to recover the function names and line numbers from the addresses listed in the \textit{Sydr}'s log file. This allows for the comparison of the output of the sanitized binary with the information from the \textit{Sydr}'s log file, thus enabling the verification of the violation.

\newparagraph{Rust Implementation}

To implement the annotation process in Rust, a \textit{dbginfo} module was developed, responsible for parsing the DWARF debug information using the \textit{gimli} and \textit{addr2line} crates. The \textit{dbginfo} is then used by the \textit{annotate} module, which is responsible for the annotation process itself.

To make the annotation process efficient, an in-memory caching mechanism was implemented to prevent parsing the debug information for the same binary more than once. This caching approach significantly improves the performance of the annotation process by reusing parsed debug information across multiple calls to the \mintinline{rust}{annotate_log(log_path, annotated_path)} function.

\subsubsection{Experimental Evaluation}

After implementing the new solution, multiple benchmarks were executed to evaluate its performance. The findings of these experiments and insights into the effectiveness and efficiency of the approach are discussed in Section \ref{results:security-invariants-enhancement-log-annotation}.

% The speedup to annotate \textit{branch trace}s was measured to be up to 99.95 \% faster. The annotation speed for \textit{instruction trace}s was on the same level measuring up to 99.13 \% speed increase.

The measurements showed that the annotation process for \textit{branch traces} was up to 99.97\% faster, while the annotation speed for \textit{instruction traces} showed a comparable improvement of up to 99.13\%.

% These performance gains demonstrate the effectiveness of our optimization approach in accelerating the analysis of both types of traces.


\section{Results}

\subsection{PyTorch Bugs} \label{results:pytorch-bugs}

Based on the findings during this project we've applied to talk at the Positive Hack Days (PHD) 2023 conference. The talk was accepted, and we presented our findings at the conference.

\subsubsection{\#91999: Case study of a bug in PyTorch}
\subsubsection{\#91999: Case study of a bug in PyTorch}

\subsection{Experimental Results for the Scheduling Strategy} \label{results:symbolic-pointers-modeling-scheduling}

To assess the quality of proposed changes we have conducted multiple experiments.

The first one: symptr-15-1

\begin{figure}[h]
    \centering
    \includegraphics[width=0.9\textwidth]{assets/fuzzbench/symptr-15-1/median-relative-code-coverage-on-each-benchmark.png}
    \caption{Median relative code coverage on each benchmark (N=15)}
    \label{fig:fuzzbench-symptr-15-1-coverage}
\end{figure}

\begin{figure}[h]
    \centering
    \includegraphics[width=0.9\textwidth]{assets/fuzzbench/symptr-15-1/avg-score-avg-rank.png}
    \caption{Average score and average rank (N=15)}
    \label{fig:fuzzbench-symptr-15-1-score-rank}
\end{figure}

The second one: symptr-25-1

\begin{figure}[h]
    \centering
    \includegraphics[width=0.9\textwidth]{assets/fuzzbench/symptr-25-1/median-relative-code-coverage-on-each-benchmark.png}
    \caption{Median relative code coverage on each benchmark (N=25)}
    \label{fig:fuzzbench-symptr-25-1-coverage}
\end{figure}

\begin{figure}[h]
    \centering
    \includegraphics[width=0.9\textwidth]{assets/fuzzbench/symptr-25-1/avg-score-avg-rank.png}
    \caption{Average score and average rank (N=25)}
    \label{fig:fuzzbench-symptr-25-1-score-rank}
\end{figure}

\newpage

And, finally, the third one: symptr-25-vs-35-2

\begin{figure}[h]
    \centering
    \includegraphics[width=0.9\textwidth]{assets/fuzzbench/symptr-25-vs-35-2/median-relative-code-coverage-on-each-benchmark.png}
    \caption{Median relative code coverage on each benchmark (N=25 vs N=35)}
    \label{fig:fuzzbench-symptr-25-vs-35-2-coverage}
\end{figure}

\begin{figure}[h]
    \centering
    \includegraphics[width=0.9\textwidth]{assets/fuzzbench/symptr-25-vs-35-2/avg-score-avg-rank.png}
    \caption{Average score and average rank (N=25 vs N=35)}
    \label{fig:fuzzbench-symptr-25-vs-35-2-score-rank}
\end{figure}

\subsection{Annotate} \label{results:annotate}

\definecolor{Gray}{gray}{0.9}
\newcolumntype{g}{>{\columncolor{Gray}}c}
\begin{table}[h]
    \centering
    \begin{tabular}{cgcgcg}
        \toprule
        \multirow{2}{*}{\bfseries Benchmark} & \multicolumn{2}{c}{\bfseries Default (ms)} & \multicolumn{2}{c}{\bfseries Rust (ms)} & \textbf{Performance diff}                            \\
                                             & Mean                                       & Std                                     & Mean                      & Std          &           \\
        cjson                                & 292.25                                     & $ \pm 66.1337 $                         & 1.08                      & $ \pm 0.06 $ & -99.63 \% \\
        libjpeg                              & 931.96                                     & $ \pm 55.3113 $                         & 1.75                      & $ \pm 0.19 $ & -99.81 \% \\
        libpng                               & 924.03                                     & $ \pm 567.277 $                         & 1.64                      & $ \pm 0.26 $ & -99.82 \% \\
        libxml2                              & 38308.54                                   & $ \pm 796.162 $                         & 12.27                     & $ \pm 0.33 $ & -99.97 \% \\
        minigzip                             & 14567.67                                   & $ \pm 121.165 $                         & 7.51                      & $ \pm 0.48 $ & -99.94 \% \\
        pcre2                                & 14562.59                                   & $ \pm 1419.99 $                         & 7.26                      & $ \pm 0.58 $ & -99.95 \% \\
        readelf                              & 19486.86                                   & $ \pm 1015.72 $                         & 7.02                      & $ \pm 0.36 $ & -99.96 \% \\
        rizin                                & 31061.09                                   & $ \pm 1800.72 $                         & 12.77                     & $ \pm 0.22 $ & -99.96 \% \\
        yices                                & 8936.02                                    & $ \pm 111.98 $                          & 3.98                      & $ \pm 0.3 $  & -99.96 \% \\
        yodl                                 & 16907.58                                   & $ \pm 377.37 $                          & 7.97                      & $ \pm 0.31 $ & -99.95 \% \\
        \bottomrule
    \end{tabular}
    \caption{Mean execution time to annotate 10 branch traces with the default and Rust annotator.}
    \label{table:annotate-branch-traces-10}
\end{table}

\definecolor{Gray}{gray}{0.9}
\newcolumntype{g}{>{\columncolor{Gray}}c}
\begin{table}[h]
    \centering
    \resizebox{\columnwidth}{!}{%
        \begin{tabular}{ccgcgcg}
            \toprule
            \multirow{2}{*}{\bfseries Benchmark} & \multirow{2}{*}{\bfseries Size (mb)} & \multicolumn{2}{c}{\bfseries Default (s)} & \multicolumn{2}{c}{\bfseries Rust (s)} & \textbf{Performance diff}                             \\
                                                 &                                      & Mean                                      & Std                                    & Mean                      & Std           &           \\
            cjson                                & 1.21                                 & 20.74                                     & $ \pm 0.05 $                           & 0.23                      & $ \pm 0.005 $ & -98.87 \% \\
            libjpeg                              & 6.69                                 & 176.9                                     & $ \pm 0.60 $                           & 2.32                      & $ \pm 0.027 $ & -98.69 \% \\
            libpng                               & 2.08                                 & 57.6                                      & $ \pm 0.36 $                           & 0.75                      & $ \pm 0.019 $ & -98.69 \% \\
            libxml2                              & 50.3                                 & timeout                                   & timeout                                & 21.27                     & $ \pm 1.016 $ & XXX \%    \\
            minigzip                             & 0.75                                 & timeout                                   & timeout                                & 166.59                    & $ \pm 1.863 $ & XXX \%    \\
            pcre2                                & 6.62                                 & 172.77                                    & $ \pm 0.69 $                           & 2.8                       & $ \pm 0.048 $ & -98.38 \% \\
            readelf                              & 9.19                                 & 454.11                                    & $ \pm 40.68 $                          & 3.94                      & $ \pm 0.144 $ & -99.13 \% \\
            rizin                                & 34.04                                & timeout                                   & timeout                                & 15.15                     & $ \pm 0.925 $ & XXX \%    \\
            yices                                & 10.48                                & 294                                       & $ \pm 1.1 $                            & 4.69                      & $ \pm 0.123 $ & -98.4 \%  \\
            yodl                                 & 22.53                                & timeout                                   & timeout                                & 10.3                      & $ \pm 1.014 $ & XXX \%    \\
            \bottomrule
        \end{tabular}
    }
    \caption{Mean execution time (over 3 runs) to annotate instruction trace with the default and Rust annotator.}
    \label{table:annotate-instruction-trace-3}
\end{table}

\section{Conclusion}

The study conducted involved a comprehensive security analysis of PyTorch utilizing hybrid fuzzing techniques. As a result of the research, various fuzzing targets were developed, and multiple previously unknown bugs were identified and reported, all of which have been subsequently fixed.

In addition, performance enhancements were made to the hybrid fuzzing tool, \textit{sydr-fuzz}, developed by ISP RAS. These improvements included optimizing the security predicates verification stage and implementing a scheduling strategy for the memory modeling mode. The experimental results of the proposed enhancements showed a significant improvement in the overall performance, thus demonstrating the effectiveness of the proposed optimizations.

Future work may include further improvements to the hybrid fuzzing tool, such as implementing a more sophisticated scheduling strategy for the memory modeling mode. Besides improving the tool itself, it is also possible to extend the study to other deep learning frameworks, such as TensorFlow.

\newpage

% 1 command to write all citations into the References
\printbibliography[heading=bibintoc]   % [] are for appearing in Table of Contents

\end{document}
